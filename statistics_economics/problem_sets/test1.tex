\documentclass{tufte-handout}

%------------------------------------------------
%\geometry{showframe} % display margins for debugging page layout
%------------------------------------------------
\usepackage{graphicx} % allow embedded images
  \setkeys{Gin}{width=\linewidth,totalheight=\textheight,keepaspectratio}
  \graphicspath{{/home/swl/Dropbox/ucd/advanced_macro/figures/}}  % set of paths to search for images
\usepackage{amsmath}  % extended mathematics
\usepackage{booktabs} % book-quality tables
\usepackage{units}    % non-stacked fractions and better unit spacing
\usepackage{multicol} % multiple column layout facilities
\usepackage{lipsum}   % filler text
\usepackage{fancyvrb} % extended verbatim environments
\usepackage{courier}
  \fvset{fontsize=\normalsize}% default font size for fancy-verbatim environments
\hypersetup{colorlinks=true, linkcolor=black, citecolor=black, urlcolor=blue}
%------------------------------------------------
% Standardize command font styles and environments
\newcommand{\doccmd}[1]{\texttt{\textbackslash#1}}% command name -- adds backslash automatically
\newcommand{\docopt}[1]{\ensuremath{\langle}\textrm{\textit{#1}}\ensuremath{\rangle}}% optional command argument
\newcommand{\docarg}[1]{\textrm{\textit{#1}}}% (required) command argument
\newcommand{\docenv}[1]{\textsf{#1}}% environment name
\newcommand{\docpkg}[1]{\texttt{#1}}% package name
\newcommand{\doccls}[1]{\texttt{#1}}% document class name
\newcommand{\docclsopt}[1]{\texttt{#1}}% document class option name
\newenvironment{docspec}{\begin{quote}\noindent}{\end{quote}}% command specification environment
%------------------------------------------------

%------------------------------------------------
%%%% Details %%%%
%------------------------------------------------
\title{Statistics for economics \\ Computer test 2}
\author{School of Economics, University College Dublin}
\date{Spring 2017} 

\begin{document}
\maketitle  

%------------------------------------------------------------------------------
\vspace{.5cm}
Download \texttt{coffeeyield.RData} from the blackboard website and load it in R. 
The data is taken from the 2002 Nature paper "The Value of Bees to the Coffee Harvest" by Roubik, and contains data on coffee yields (in $kg/ha^{-1}$). 
It includes the following variables
\begin{itemize}
  \item \texttt{country}, country name
  \item \texttt{world}, factor variable indicating whether the country belongs to the old or new world
  \item \texttt{world.d}, binary indicator whether the country belongs to the old (0) or new world (1)
  \item \texttt{yield61to80}, coffee yield between 1961-1980
  \item \texttt{yield81to01}, coffee yield between 1961-1980
  \item \texttt{ond}, the temperature anomaly for the months October, November, December
  \item \texttt{ond.above}, a binary indicator for above average temperature for October-December
\end{itemize}

Use the data to answer the questions in the next section. 
Each question is worth one point for a total of 10 points. 
Write your answers down in your favourite word processor and include the figures you produce. 
When you're done you need to send the document with your answers to \href{stijn.vanweezel@ucd.ie}{stijn.vanweezel@ucd.ie}

\clearpage
%------------------------------------------------------------------------------
\section{Questions}
\begin{enumerate}
  \item What is the average and standard deviation of the temperature anomaly?
  \item Use a boxplot to visualise the distribution of the temperature anomaly. Describe the distribution.
  \item Produce a line plot of the temperature anomalies over time. What does the data show?
  \item How many El Ni\~no years are there in total? What percentage of years is an El Ni\~no year?
  \item How many years experienced at least a strong El Ni\~no? 
  \item If there is an El Ni\~no, what is the probability that it is a very strong El Ni\~no?
  \item If there was an El Ni\~no last year, what is the probability that there will not be an El Ni\~no this year?
  \item Use a boxplot to plot the anomaly for the temperature between October-December against the strength of the El Ni\~no. Which conclusions can you draw from the data?
  \item What is the probability that a year with higher than average OND temperatures does not correspond with an El Ni\~no?
  \item What is the probability that a year with below average temperatures for October-December experiences a moderate to strong El Ni\~no?
\end{enumerate}
\end{document}
