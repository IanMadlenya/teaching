\documentclass{beamer}
\usetheme{}
\usecolortheme{dolphin}           
\useinnertheme{circles}
\setbeamertemplate{itemize items}[default]
\setbeamertemplate{enumerate items}[default]
\usepackage[T1]{fontenc}
\usepackage[utf8]{inputenc}
\usepackage{lmodern}
\usepackage{amsmath}
\usepackage{booktabs} 
\usepackage{graphicx}        
\usepackage{array}
\usepackage{color}
\usepackage{svg}
\makeatletter
\def\zapcolorreset{\let\reset@color\relax\ignorespaces}
\def\colorrows#1{\noalign{\aftergroup\zapcolorreset#1}\ignorespaces}
\makeatother
\graphicspath{{/home/swl/Dropbox/ucd/eu_economics/figs/}} 
\setbeamertemplate{navigation symbols}{}

%--------------------------------------
%%%% DETAILS TITLE PAGE %%%%
%--------------------------------------
\title{European Economy:\\ Summary}
\author{School of Economics, University College Dublin}
\date{Spring 2017}
\begin{document}
%--------------------------------------
%%%% TITLE SLIDE %%%%
%--------------------------------------
\begin{frame}
\titlepage  
\end{frame}

%--------------------------------------
\begin{frame}{Optimum Currency Area Theory}
  \begin{itemize}
    \item Benefits 
      \begin{enumerate}
        \item Lower transaction costs
        \item Price transparency
        \item Uncertainty reduction
        \item Improvements in trade
        \item Quality of monetary policy
      \end{enumerate}
      \medskip
    \item Costs
    \begin{enumerate}
      \item Link between shocks and exchange rate
      \item Asymmetric shocks
      \item Symmetric shocks with asymmetric effects
    \end{enumerate}
  \end{itemize}
\end{frame}

%--------------------------------------
\begin{frame}{Criteria for a common currency area}\framesubtitle{NB - These criteria are endogenous}
  \begin{enumerate}
    \item Labour mobility
    \item Production diversification
    \item Trade openness
    \medskip
    \item Fiscal transfers
    \item Homogeneous preferences
    \item Solidarity
  \end{enumerate}  
\end{frame}

%--------------------------------------
\begin{frame}{Is the EU and optimum currency area?}
  \begin{table}[!h] \centering \label{table:summary}
\scalebox{1}{\begin{tabular}{lc}
\\[-1.8ex]\hline 
\hline \\[-1.8ex] 
Criterion & Satisfied\\
\hline \\[-1.8ex]\\
Labour mobility & No\\
Trade openness  & Yes\\
Product diversification & Yes\\
Fiscal transfers & No\\
Homogeneous preferences & Partially\\
Commonality of destiny & Hard to tell\\
    \\[-1.8ex]\hline 
    \hline \\[-1.8ex]
\end{tabular} }  
\end{table}
\end{frame}

%--------------------------------------
\begin{frame}{Main principles of the Economic and Monetary Union (EMU)}
  \begin{enumerate}
    \item Provide price stability
    \begin{itemize}
      \item Keep inflation between 1.5-2\% over the medium term
    \end{itemize}
    \medskip
    \item Central bank independence
    \begin{itemize}
      \item European Central Bank's main aim is securing price stability
    \end{itemize}
    \medskip
    \item Fiscal discipline
    \begin{itemize}
      \item Budget deficit below 3\% of GDP
      \item Public debt should not exceed 60\% of GDP
    \end{itemize}
  \end{enumerate}
\end{frame}

%--------------------------------------
\begin{frame}{Fiscal policy in the EU}
  \begin{itemize}
    \item Fiscal policy set by member states, but subject to Maastricht Treaty
    \medskip
    \item Remaining macroeconomic instrument to deal with shocks
    \begin{itemize}
      \item Slow to implement
      \item Budget needs to be balanced
      \item Fiscal policy is countercyclical
    \end{itemize}
    \medskip
    \item Enforcement of debt/deficit limits through Stability and Growth Pact
    \medskip
    \item Fiscal compliance with Maastricht Treaty is poor
    \begin{itemize}
      \item Except for most new member states
      \item Even with compliance things can go wrong, e.g. Spain
    \end{itemize}
  \end{itemize}
\end{frame}

%--------------------------------------
\begin{frame}{The Eurocrisis}
  \begin{itemize}
    \item Large macroeconomic shock with asymmetric effects
    \begin{enumerate}
      \item Exposed euro design faults
      \item Underwhelming response EU
    \end{enumerate}
    \medskip
    \item Effects Great Recession amplified by 
    \begin{enumerate}
      \item High levels government debt
      \item Financial integration (debt exposure)
      \item Trade imbalances
    \end{enumerate}
    \medskip
    \item Illustrated competitiveness problem of EU countries
  \end{itemize}
\end{frame}

%--------------------------------------
\begin{frame}{The Eurocrisis}
  \begin{itemize}
    \item EU response complicated by politics
    \begin{itemize}
      \item Little agreement on how to deal with issue
      \item Ad hoc measures (financial bail-outs), combined with harsh austerity
      \item No cyclical fiscal transfers
    \end{itemize}
    \medskip
    \item ECB response very conservative
    \begin{itemize}
      \item Discussion about mandate
      \item More hands on since Draghi
    \end{itemize}
    \medskip
    \item Recovery takes long time for certain countries
    \begin{itemize}
      \item Economic divergence
      \item Need to implement structural reforms
    \end{itemize}
  \end{itemize}
\end{frame}

%--------------------------------------
\begin{frame}{Economic growth in the EU}
  \begin{itemize}
    \item Slow growth since 1990s 
    \begin{itemize}
      \item Growth mainly through capital deepening
      \item Little improvement in total factor productivity
    \end{itemize}
    \medskip
    \item Productivity slump potentially result of
    \begin{enumerate}
      \item Taxation level
      \item Regulations
      \item Level of competition
    \end{enumerate}
    \medskip
    \item Main macroeconomic issues in EU
    \begin{enumerate}
      \item Lack of competitiveness
      \item Low employment rates
      \item Gloomy demographic prospects
    \end{enumerate}
    \medskip
    \item Potential growth stimulants of EU membership
    \begin{enumerate}
      \item Boost in trade and investment
      \item Lower consumer prices
      \item Improvement in fiscal and monetary policy
    \end{enumerate}
    \medskip
    \item New members could potentially benefit from joining the Euro
  \end{itemize}
\end{frame}


%------------------------------------------------------------------------------
\end{document}
