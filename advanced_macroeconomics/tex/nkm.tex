\documentclass{beamer}
\usetheme{}
\usecolortheme{dolphin}           
\useinnertheme{circles}
\setbeamertemplate{itemize items}[default]
\setbeamertemplate{enumerate items}[default]
\usepackage[T1]{fontenc}
\usepackage[utf8]{inputenc}
\usepackage{lmodern}
\usepackage{amsmath}
\usepackage{booktabs} 
\usepackage{graphicx}        
\usepackage{array}
\usepackage{color}
\makeatletter
\def\zapcolorreset{\let\reset@color\relax\ignorespaces}
\def\colorrows#1{\noalign{\aftergroup\zapcolorreset#1}\ignorespaces}
\makeatother
\graphicspath{{/home/swl/Dropbox/ucd/advanced_macro/figures/}} 
\setbeamertemplate{navigation symbols}{}
\setbeamertemplate{footline}[frame number]

%--------------------------------------
\title{New Keynesian model}
\author{School of Economics, University College Dublin}
\date{Spring 2018}
\begin{document}

%--------------------------------------
\begin{frame}
 \titlepage
\end{frame}
%--------------------------------------

%--------------------------------------
\begin{frame}
  \textbf{New Keynesian model} addresses some of the critiques on the Keynesian model
  \begin{itemize}
    \item Rational expectations
    \item People behave optimally
  \end{itemize}
  \medskip
  Also room for monetary policy to have systematic effects
  \begin{itemize}
    \item Central mechanism for monetary policy is \textbf{sticky prices}
    \item If prices don't move in line with money; central bank can't control real money supply or interest rate
  \end{itemize}
\end{frame}
%--------------------------------------

%--------------------------------------
\begin{frame}
 \textbf{Dixit-Stiglitz model}\\
 Used as vantage point to describe optimal behaviour  

\begin{align}
  Y_t=\left( \int_0^1 Y_t(i)^{\frac{\theta-1}{\theta}}di\right)^{\frac{\theta}{\theta-1}}
\end{align}
Consumers will maximise their utility function $U(Y_t)$ over an aggregate of a continuum of differentiated goods
\begin{itemize}
  \item $\theta$ denotes constant elasticity of substitution
\end{itemize}
\medskip
This model does not include capital but only consumption goods. 
\end{frame}
%--------------------------------------


%--------------------------------------
\begin{frame}
 Demand function for each differentiated good is of the form
\begin{align}
  Y_t(i)=Y_t \left( \frac{P_t(i)}{P_t}\right)^{-\theta}
\end{align}
\medskip
$P_t$ is the aggregate price index which is defined by
\begin{align}
  P_t=\left( \int_o^1 P_t(i)^{1-\theta}di \right)^{\frac{1}{1-\theta}}
\end{align}  
\end{frame}
%--------------------------------------

%--------------------------------------
\begin{frame}
 \textbf{Calvo model}\\
 Used to describe price rigitidy, or sticky prices
\begin{align}
  P_t &= \left[(1-\alpha)X_t^{1-\theta} + \alpha P_{t-1}^{1-\theta} \right] ^{\frac{1}{1-\theta}}\\
  P_t^{1-\theta} &= (1-\alpha)X_t^{1-\theta} + \alpha P_{t-1}^{1-\theta}
\end{align}
$1-\alpha$ is random fraction of firms able to reset their price\\
$X_t$ is the price that the firms resetting today have chosen
\begin{enumerate}
  \item All other firms keep prices unchaged
  \item All firms setting new prices today set the same price.
\end{enumerate}
Apart from this difference in timing, of when they set prices, the firms are completely symmetric
\end{frame}
%--------------------------------------

%--------------------------------------
\begin{frame}
 Prices may be fixed for many periods: Firms pick a price to maximise
\begin{align}
  E_t \left[ \sum_{k=0}^{\infty} (\alpha \beta)^k (Y_{t+k}P_{t+k}^{\theta-1}X_t^{1-\theta} -
  P_{t+l}^{-1}C (Y_{t+k}P_{t+k}^{\theta}X_t^{-\theta}) \right]
\end{align}
$C(.)$ is the cost function\\
Solution for maximisation problem is (differentiating with regard to $X_t$)
\begin{align}
  X_t = \frac{\theta}{\theta-1} \frac{E_t \left(\sum_{k=0}^{\infty}(\alpha \beta)^k Y_{t+k}P_{t+k}^{\theta-1}MC_{t+k} \right)}
  {E_t \left(\sum_{k=0}^{\infty}(\alpha \beta)^k Y_{t+k}P_{t+k}^{\theta-1} \right) }
\end{align}  
\end{frame}
%--------------------------------------

%--------------------------------------
\begin{frame}
  This entails that changed price $X_t$ is a markup over a weighted average of future marginal costs: without frictions the firm would set 
  \begin{align}
    X_t=\frac{\theta}{\theta-1}MC_t
  \end{align}
i.e. the price equals $\frac{\theta}{\theta-1}$ times the marginal costs\\
Got two non-linear equations for price
\begin{align}
    P_t^{1-\theta} &= (1-\alpha)X_t^{1-\theta} + \alpha P_{t-1}^{1-\theta}\\
    X_t &= \frac{\theta}{\theta-1} \frac{E_t \left(\sum_{k=0}^{\infty}(\alpha \beta)^k Y_{t+k}P{t+k}^{\theta-1}MC_{t+k} \right)}
  {E_t \left(\sum_{k=0}^{\infty}(\alpha \beta)^k Y_{t+k}P{t+k}^{\theta-1} \right) }
\end{align}
\end{frame}
%--------------------------------------

%--------------------------------------
\begin{frame}
  Solving or simulating price equations is not easy: use log-linear approximations taken around constant growth, zero inflation path
  \begin{align}
    X_t^* &= P^*=P^*_{t-1}=P^*\\
  X^* &= \left(\frac{\theta}{1-\theta}\right)MC^*    
  \end{align}
\begin{align}
  p_t &= (1-\alpha)x_t + \alpha p_{t-1}\\
  x_t &= (1-\alpha \beta) \sum_{k=0}^{\infty} (\alpha \beta)^k E_tmc_{t+k}
\end{align} 
\end{frame}
%--------------------------------------

%--------------------------------------
\begin{frame}
  Reverse engineering can see that optimal reset price can be written as
  \begin{align}
  x_t=(1-\alpha \beta) mc_t + (\alpha \beta) E_t x_{t+1}
 \end{align}
 Can combine this with fact that 
\begin{align}
  p_t &= (1-\alpha)x_t+\alpha p_{t-1}\\
  x_t &= \frac{1}{1-\alpha}(p_t-\alpha p_{t-1})  
\end{align}
\end{frame}
%--------------------------------------

%--------------------------------------
\begin{frame}
  Note that 
  \begin{align}
    \pi_t=p_t-p_{t-1}
  \end{align}
  After bunch of re-arranging you get  
\begin{align}
  \pi_t= \beta E_t \pi_{t+1} + \frac{(1+\alpha)(1-\alpha \beta)}{\alpha}(mc_t - p_t)
\end{align}
$\pi_t$ is a function 
\begin{enumerate}
  \item Expected inflation in $t+1$
  \item Ratio of marginal cost to price (real marginal cost)
\end{enumerate}
  This is the \textbf{New-Keynesian Phillips curve} (NKPC)
\end{frame}
%--------------------------------------

%--------------------------------------
\begin{frame}
  \textbf{Output}\\
  Assume that there are standard diminishing returns to labour production function
  \begin{itemize}
    \item Higher output reduces marginal productivity and raises marginal cost 
  \end{itemize}
  This means that real marginal costs are a function of the output gap
\begin{align}
  mc_t-p_t=\eta x_t
\end{align}
Concerning $x_t$
\begin{align}
  x_t=y_t-y_t^n
\end{align}
$y_t^n$ is the path of output that would have been obtained in a zero inflation price friction free economy
\end{frame}
%--------------------------------------

%--------------------------------------
\begin{frame}
 We get NKPC of the form
  \begin{align}
  \pi_t=\beta E_t \pi_{t+1} + \kappa x_t
\end{align}
\medskip
 Looks a lot like traditional expectations-augmented Phillips curve
It is a first-order stochastic difference equation, which entails a solution in the form
\begin{align}  
  \pi_t=\kappa \sum_{k=0}^{\infty}\beta^k E_t x_{t+k} 
\end{align}
\end{frame}
%--------------------------------------

%--------------------------------------
\begin{frame}
\begin{align}  
  \pi_t=\kappa \sum_{k=0}^{\infty}\beta^k E_t x_{t+k} 
\end{align}
\medskip
Has no backward-looking element
\begin{itemize}
  \item No intrinsic inertia in inflation
  \item Lagged inflation effects, in conventional models, are actually a statistical artifact
\end{itemize}
\medskip
Note that original formulation of NKPC does not have error/shock term
\begin{itemize}
  \item Maybe price movements not consistent with this formulation.
\end{itemize}
\begin{align}
  \pi_t=\beta E_t \pi_{t+1} + \kappa x_t + u_t
\end{align} 
 \textbf{Cost-push shock} added to NKPC
\end{frame}
%--------------------------------------

%--------------------------------------
\begin{frame}
  \begin{align}
  \pi_t=\beta E_t \pi_{t+1} + \kappa x_t + u_t
\end{align} 
 $u_t$ accounts for misc. shocks:
 \begin{itemize}
   \item $\pi$ no longer results of just expected inflation and output gap
 \end{itemize}
 \medskip
 Central bank can no longer implement a stabilisation policy by only addressing the output gap. 
\end{frame}
%--------------------------------------

%--------------------------------------
\begin{frame}
  NKPC links inflation to output: Consider how we link output to monetary policy. 
  \begin{itemize}
    \item First of three equations that make up the New Keynesian model
  \end{itemize}
  NKM uses interest rates: recall that model does not include capital
  \begin{align}
    Y=C
  \end{align}
   Relation between $C$ and $i$ comes from standard intertemporal optimization problem; consumer wants to maximise
\begin{align}  
 \sum_{k=0}^{\infty}\left(\frac{1}{1+\beta}\right)^k U(C_{t+k}) 
 \end{align}  
\end{frame}
%--------------------------------------

%--------------------------------------
\begin{frame}
  Intertemporal budget constraint given by  
\begin{align}
  \sum_{k=0}^{\infty} \frac{E_t C_{t+k}}{\left(\prod_{m=1}^{k+t}R_{t+m} \right)} =
   A_t +   \sum_{k=0}^{\infty} \frac{E_t Y_{t+k}}{\left(\prod_{m=1}^{k+t}R_{t+m} \right)}
\end{align}
$R_t$ is the interest rate\\
Can write Lagrangian as
\begin{align}
  L &= \sum_{k=0}^{\infty} \left(\frac{1}{1+\beta} \right)^k U(C_{t+k}) \\ \nonumber
  &+ \lambda \left[A_t + \sum_{k=0}^{\infty} \frac{E_t Y_{t+k}}{\left( \prod_{m=1}^{k+1} R_{t+m} \right)} -
  \sum_{k=0}^{\infty} \frac{E_t C_{t+k}}{\left( \prod_{m=1}^{k+1} R_{t+m} \right)} \right ]
\end{align}
\end{frame}
%--------------------------------------

%--------------------------------------
\begin{frame}
  \textbf{Euler equation} can be derived by combining FOCs for $C_t$ and $C_{t+1}$ 
\begin{align}  
  U'(C_t) = E_t \left[ \left(\frac{R_{t+1}}{1+\beta} \right) U'(C_{t+1})\right] 
  \end{align}
  Can set
  \begin{align}
    U(C_t)=U(Y_t)=\frac{Y_t^{1-\frac{1}{\sigma}}}{1-\frac{1}{\sigma}}
  \end{align}
  \begin{align}
  E_t \left[ \left( \frac{R_{t+1}}{1+\beta} \right) \left( \frac{Y_t}{Y_{t+1}} \right)^{\frac{1}{\sigma}} \right]=1 
 \end{align}
 \textbf{NB-}Similar to the Real Business Cycle model this is a Constant Relative Risk Aversion (CRRA) utility from consumption
\end{frame}
%--------------------------------------

%--------------------------------------
\begin{frame}
  Set 
  \begin{align}
    \rho=-log\beta
  \end{align}
Log-linearised version of the Euler equation is 
\begin{align}  
y_t=E_ty_{t+1} - \sigma(i_t - E_t\pi_{t+1} - \rho) 
\end{align}
i.e. today's output depends negatively on the real interest rate
\end{frame}
%--------------------------------------



%--------------------------------------
\begin{frame}
  Recall that the equation for inflation featured the output gap 
  \begin{align}
    x_t=y_t-y_t^n  
  \end{align}
  Substitute into Euler equation\footnote{$E_ty_{t+1}$ with $E_tx_{t+1}+E_ty^n_{t+1}$}
\begin{align}
  x_t &= E_t x_{t+1} - \sigma (i_t - E_t \pi_{t+1} - \rho) + E_t y_{t+1}^n - y_t^n\\
  x_t &= E_t x_{t+1} - \sigma (i_t - E_t \pi_{t+1} - r_t^n)
\end{align}
Natural interest rate is given by
\begin{align}
  r_t^n=\sigma^{-1}E_t\Delta y_{t+1}^n - log \beta  
\end{align}
\end{frame}
%--------------------------------------

%--------------------------------------
\begin{frame}
  $r_t$ is a function of
  \begin{align}
    E_t \Delta y_{t+1}^n
  \end{align}
  Meaning that it is determined by 
  \begin{enumerate}
    \item Technology
    \item Preferences
  \end{enumerate}
  Output gap $x_t$ follows a first-order stochastic difference equation which has a solution of the form
\begin{align}
  x_t = \sigma \sum_{k=0}^{\infty} (i_{t+k} - E_t \pi_{t+k+1} - r_{t+k}^n)  
\end{align}  
\end{frame}
%--------------------------------------

%--------------------------------------
\begin{frame}
  \textbf{Policy implications}\\
   $x_t$ has no backward-looking element; output has no intrinsic persistence\\
   For monetary policy this means that what matters for today's output is
   \begin{enumerate}
     \item Current policy
     \item All future interest rates
   \end{enumerate}
  \medskip
  Central bankers should therefore take care in managing expectations about future policy
  \begin{itemize}
    \item Future interest rates are their key tool
  \end{itemize}
Interpreting $i_t$ as the short-term interest rate, and assuming that the expectations theory of the term structure holds, this model states that it is the long-term interest rates that matter for spending. 

\end{frame}
%--------------------------------------

%--------------------------------------
\begin{frame}
  In its most basic form the New Keynesian model has three equations. 
We have already derived 
\begin{enumerate}
  \item The New Keynesian Phillips curve
  \begin{align} \pi_t=\beta E_t \pi_{t+1} + \kappa x_t + u_t \end{align}
  \item The Euler equation for output
  \begin{align} x_t =E_t x_{t+1} - \sigma (i_t - E_t \pi_{t+1} - r_t^n) \end{align}
\end{enumerate}
Now we only have to find an equation describing how interest rate policy is set. 
This is usually described as an explicit interest rate rule.  
\end{frame}
%--------------------------------------

%--------------------------------------
\begin{frame}
  \textbf{Output-inflation dynamics}
\begin{align}
  x_t &=E_tx_{t+1} - \sigma(i_t-E_t\pi_{t+1} - r_t^n)\\
  \pi_t &= \beta E_t \pi_{t+1} + \kappa x_t + u_t
\end{align}
  Can be rewritten as 
\begin{align}
  \pi_t = \beta E_t \pi_{t+1} + \kappa E_t x_{t+1} - \kappa\sigma(i_t - E_t \pi_{t+1} - r_t^n) + u_t
\end{align}
 Put in vector form
\begin{align}
  \begin{pmatrix} x_t \\ \pi_t \end{pmatrix} = \begin{pmatrix} 1 & \sigma \\ \kappa & \beta +\kappa\sigma \end{pmatrix}
  \begin{pmatrix} E_tx_{t+1} \\ E_t \pi{t+1} \end{pmatrix} + 
  \begin{pmatrix} \sigma (r_t^n-i_t) \\ \kappa\sigma (r_t^n-i_t) + u_t \end{pmatrix}
\end{align}  
\end{frame}
%--------------------------------------

%--------------------------------------
\begin{frame}
 Have model in form
 \begin{align}
  Z_t=AE_tZ_{t+1}+BV_t 
 \end{align}
 For unique stable solution the eigenvalues of $A$ need to be less than 1
\begin{align} 
 A=\begin{pmatrix} 1 & \sigma \\ \kappa & \beta +\kappa\sigma \end{pmatrix} 
 \end{align}
 Recall that there is an eigenvector that when multiplied by $A-\lambda I$ equals a vector of zeroes, meaning that the determinants of the matrix equal zero
\end{frame}
%--------------------------------------

%--------------------------------------
\begin{frame}
  \begin{align}
  A-\lambda I = \begin{pmatrix}
    1-\lambda & \sigma \\
    \kappa    & \beta + \kappa \sigma - \lambda
  \end{pmatrix}
 \end{align}
 Eigenvalues satisfy
\begin{align} 
    P(\lambda) &=(1-\lambda)(\beta+\kappa\sigma-\lambda)-\kappa\sigma=0\\
    P(\lambda) &= \lambda^2  -(1+\beta+\kappa\sigma)\lambda+\beta=0
\end{align} 
\end{frame}
%--------------------------------------

%--------------------------------------
\begin{frame} $P(\lambda)$ is a U-shaped polynomial: if $\lambda=0$ we get 
 \begin{align}
   P(0)=\beta>0\\
   P(1)=-\kappa \sigma <0
 \end{align}
 $P(\lambda)$ will be greater than 0 when $\lambda$ rises above one, this implies that 
 \begin{enumerate}
   \item one eigenvalue between zero and one 
   \item one eigenvalue greater than 1
 \end{enumerate}
This is a serious problem for the model: no unique stable solution; model has multiple equilibria
\end{frame}
%--------------------------------------

%--------------------------------------
\begin{frame}
  Two ways to deal with $\lambda$ issue
\begin{enumerate}
  \item Accept that there are multiple equilibria: analyse the impact of interest rate changes on output and inflation across a range of different possible equilibria
  \item Specify that monetary policy follows a particular rule and this rule is designed to produce a unique stable equilibrium
\end{enumerate}
\end{frame}
%--------------------------------------

%--------------------------------------
\begin{frame}
  \textbf{Taylor rule}  
\begin{align} 
  i_t=r_t^n+ \phi_{\pi}\pi_t+\phi_xx_t 
\end{align}
  Monetary policy sets interest rate based on inflation and output gap
  \begin{itemize}
    \item Increase in $\pi,x$ will increase $i$    
  \end{itemize}
 Note inclusion of natural interest rate: set interest rate moves with the natural interest rate
 \begin{itemize}
   \item Rule here allows $i$ to move with natural rate: Taylor's rule has a constant intercept
 \end{itemize}
\end{frame}
%--------------------------------------

%--------------------------------------
\begin{frame}
  Rule can be substituted in the equation for $x_t$ to give
\begin{align}  
  x_t=E_tx_{t+1} + \sigma E_t \pi_{t+1} - \sigma\phi_{\pi}\pi_t -\sigma\pi_x x_t 
\end{align}  
 To look at dynamics rewrite equations in matrix form
 \begin{align}
  Z_t&=\begin{pmatrix} x_t \\ \pi_t \end{pmatrix}; V_t=\begin{pmatrix}  0 \\u_t \end{pmatrix} \\
  Z_t&=AE_tZ_{t+1}+BV_t 
 \end{align}
\end{frame}
%--------------------------------------


%--------------------------------------
\begin{frame}
 In standard model
 \begin{align*}
   Z_t=AE_tZ_{t+1}+BV_t 
 \end{align*}
  We have
  \begin{align}
  A &= \frac{1}{1+\sigma\pi_x + \kappa\sigma\phi_{\pi}} \begin{pmatrix}
    1 & \sigma(1-\beta\phi_{\pi}) \\
    \kappa & \beta + \sigma\kappa + \beta(1+\sigma\phi_x)
      \end{pmatrix}\\
  B &= \frac{1}{1+\sigma\phi_x + \kappa\sigma\phi_{\pi}} \begin{pmatrix}
    1 & -\sigma\phi_{\pi}\\
    \kappa & 1+\sigma\phi_x
  \end{pmatrix}
\end{align}  
\end{frame}
%--------------------------------------

%--------------------------------------
\begin{frame}
  System is a matrix version of the first-order stochastic difference equations: can be solved in a similar fashion to give
\begin{align}
  Z_t=\sum_{k=0}^{\infty}A^k BE_tV_{t+k}
\end{align}
To have unique stable equilibrium the absolute values of both eigenvalues of $A$ need to be less than 1, which will be the case when
This will be the case when 
\begin{align}  
  \phi_{\pi}+\frac{(1-\beta)\phi_x}{\kappa}>1 
\end{align}
$\beta \approx 1$ so the condition is approximately $\phi_{\pi}>1$
\end{frame}
%--------------------------------------

%--------------------------------------
\begin{frame}
  If the policy rule satisfies this requirement, known as the \textbf{Taylor principle}, there is a unique stable equilibrium
  \begin{itemize}
    \item Nominal interest rates must rise by more than inflation so that real rates rise in response to an increase in inflation
    \item Needed for stability because otherwise inflationary shocks reduces real interest rates which stimulates the economy which will further stimulate inflation
  \end{itemize}
  A big question for central banks of course is what is optimal to do? 
  In general we know that central banks 
\begin{itemize}
  \item Don't like inflation\footnote{Ask the Germans}
  \item Like to keep output an a steady path close to potential
\end{itemize}
\end{frame}
%--------------------------------------

%--------------------------------------
\begin{frame}
  \textbf{Loss function}\\
   Central bank behaviour can be modeled using loss function   
  \begin{align} 
  L_t = \frac{1}{2}\sum_{t=0}^{\infty}\beta^tE_t(\pi_{t+k}^2 + \gamma x_{t+k}^2) 
\end{align}
$x_t$ is the output gap\\
$\gamma$ indicates the weight put on stabilisation relative to inflation stabilisation\\
$\kappa$ is the coefficient on the output gap in the NKPC \\
$\theta$ is the elasticity of demand for firms\\

\end{frame}
%--------------------------------------

\begin{frame}
Quadratic loss functions like these are popular since differentiating things to the power 2 produces linear relationships
 \begin{itemize}
   \item Quadratic loss function can also be used as an approximation to consumer utility in the NKM
 \end{itemize}
 Research has shown that 
 \begin{align}
   \gamma=\frac{\kappa}{\theta}   
 \end{align}
\end{frame}




%--------------------------------------
\begin{frame}
  \begin{align}
    x_t^2 
  \end{align}
  Risk-averse consumers prefer smooth consumption paths which keeps output close to its natural rate to achieve this.
  \begin{align}
    \pi_t^2
  \end{align}
  Consumers don't just care about the level of consumption but also its allocation. 
  \begin{itemize}
    \item With inflation, sticky prices imply different prices for the symmetric goods and thus different consumption levels
    \item Optimality requires equal consumption of all items in the bundle. Rationale for welfare effect of inflation, independent of its effect on output
  \end{itemize}
\end{frame}
%--------------------------------------

%--------------------------------------
\begin{frame}
  \textbf{Optimal policy under commitment}\\
  Suppose that the central bank can commit today to a strategy it can adopt now and in the future. 
\begin{align}
  L= \sum_{t=0}^{\infty} \beta^tE_t \left[ \frac{1}{2}  (\pi_{t+k}^2 + \gamma x_{t+k}^2) + 
  \lambda_{t+k} (\pi_{t+k} - \beta \pi_{t+k+1} - \kappa x_{t+k}) \right] 
\end{align}
FOCs are 
\begin{align}
  \gamma E_t x_{t+k} - \kappa E_t \lambda_{t+k} &= 0\\
  E_t\pi_{t+k} + E_t\lambda_{t+k} - E_t \lambda_{t+k-1} &=0
\end{align}
for $t=0,1,2,...$ where $\lambda_{-1}=0$
\begin{itemize}
  \item There is no constraint on time $t=-1$
\end{itemize}
\end{frame}
%--------------------------------------

%--------------------------------------
\begin{frame}
  From this we get that 
\begin{align}
  E_t x_{t+k}= \frac{\kappa}{\gamma} E_t\lambda_{t+k} &= \theta E_t \lambda_{t+k}\\
  E_t \pi_{t+k} = E_t \lambda_{t+k-1} - E_t \lambda_{t+k} &= -\frac{1}{\theta}E_t \Delta x_{t+k}\\
  \Delta E_t x_{t+k} &= - \theta E_t \pi_{t+k}
\end{align}

Therefore the optimal policy under commitment will be characterised by
\begin{align}
  x_t &= -\theta\pi_t = \theta(p_{t-1}-p_t)\\
  E_t \Delta x_{t+1} &= -\theta E_t \pi_{t+k} = \theta (p_{t+k-1} - p_{t+k})
\end{align}  
\end{frame}
%--------------------------------------

%--------------------------------------
\begin{frame}
  If we consider some initial price level $p_{-1}$ we get
\begin{align}  
  E_t x_{t+k} = \theta(p_{-1} - E_tp_{t+k}) 
\end{align}
Since
\begin{align}
  \pi_t=p_t-p_{t-1}
\end{align}
\end{frame}
%--------------------------------------

%--------------------------------------
\begin{frame}
  \begin{align}  
  E_t x_{t+k} = \theta(p_{-1} - E_tp_{t+k}) 
\end{align}
Optimal policy is set against the price level
\begin{itemize}
  \item Shocks will will only temporarily affect price level but have no cumulative effect
  \item On average inflation will be zero 
\end{itemize}
\medskip
\textbf{NB-} Policy is history dependent: policy today depends on the whole past sequence of shocks that have determined today's price level
\end{frame}
%--------------------------------------


%--------------------------------------
\begin{frame}
  \textbf{Optimal policy under discretion}\\  
  Consider scenario where a central bank cannot commit to taking a particular course of action in the future
  \begin{itemize}
    \item All they can do is adopt the optimal strategy for what to do today.  
  \end{itemize}
Recall that the optimality conditions for period $t$ and $t+1$ are
\begin{align}
  x_t &= -\theta\pi_t\\
  E_tx_t - E_tx_{t+1} &= -\theta\pi_{t+1}
\end{align}
The conditions for the first period are different from the rest
\begin{itemize}
   \item At $t$, $t-1$ is gone and doesn't matter
   \item Have to take into account the effect that time $t$ decisions will have at time $t+1$
 \end{itemize} 
\end{frame}
%--------------------------------------

%--------------------------------------
\begin{frame}
 Under discretion the policy maker always sets 
\begin{align}
  x_t=-\theta\pi_t
\end{align}
Policy is set against inflation, where inflation can be characterised by 
\begin{align}
  \pi_t=\beta E_t\pi_{t+1}-\kappa\theta\pi_t+u_t
\end{align}
With new first-order difference equation
\begin{align}
  \pi_t=\left(\frac{1}{1+\theta\kappa}\right) (\beta E_t\pi_{t+1} + u_t)
\end{align}
and a repeated iteration solution
\begin{align}
  \pi_t = \left(\frac{1}{1+\theta\kappa}\right) \sum_{k=0}^{\infty} \left(\frac{\beta}{1+\theta\kappa}\right)^k E_tu_{t+k}
\end{align}
\end{frame}
%--------------------------------------

%--------------------------------------
\begin{frame}
  It is often assumed that the cost-push shocks follow an $AR(1)$ process which implies 
  \begin{align}
    E_tu_{t+k}=\rho^ku_t
  \end{align}
  With 
  \begin{align}
 u_t=\rho u_{t-1} + v_t,\; v_t \sim N(0,\sigma^2) 
\end{align}
\end{frame}
%--------------------------------------

%--------------------------------------
\begin{frame}
  Using
  \begin{align}
    \sum_{k=0}^{\infty}c^k=\frac{1}{1-c}
  \end{align}
for $|c|<1$, inflation becomes
\begin{align}
  \pi_t &= \left( \frac{1}{1-\theta\kappa} \right) \left[ \sum_{k=0}^{\infty} \left( \frac{\beta\rho}{1+\theta\kappa} \right)^k \right]u_t\\
  &= \left( \frac{1}{1-\theta\kappa} \right) \left( \frac{1}{1-\frac{\beta\rho}{1+\theta\kappa}} \right) u_t\\
  &= \frac{u_t}{1+\theta\kappa-\beta\rho}
\end{align}
\end{frame}
%--------------------------------------%--------------------------------------
\begin{frame}
  The $AR(1)$ cost-push shock implies that 
\begin{align}
  E_tx_{t+1} &= \rho x_t\\
  E_t \pi_{t+1} &= \rho \pi_t  
\end{align}
which can be substituted in the Euler equation along with 
\begin{align}
  x_t&=-\theta\pi_t\\ \nonumber
  x_t&=E_t x_{t+1}-\sigma(i_t-E_t \pi_{t+1}-r_t^n)
\end{align}
to back out what the optimal interest rate looks like  
\end{frame}
%--------------------------------------

%--------------------------------------
\begin{frame}
  \begin{align}
  i_t=r_t^n+ \left( \rho + \frac{(1-\rho)\theta}{\sigma} \right) \pi_t
\end{align}
This will be greater than 1 if 
\begin{align}
 \frac{\theta}{\sigma}>1 
\end{align}
which will hold for all reasonable parameterisations
\begin{itemize}
   \item Inflation and thus interest rates do not depend at all on what happened in the past.
 \end{itemize} 
\end{frame}
%--------------------------------------

%--------------------------------------
\begin{frame}
  \textbf{Woodford} (2003) argues that policy under commitment produces superior welfare outcomes
  \begin{itemize}
    \item Private sector will anticipate future policies will be different
    \item Conditions at time $t$ have the potential to improve stabilisation outcomes at time $t+1$
    \item Holds even if these conditions will actually no longer matter at a later time. 
  \end{itemize}
\end{frame}
%--------------------------------------

%--------------------------------------
\begin{frame}
  Have to consider the transitory cost-push shock $u_t$; Woodford argues expectations about shock won't affect future policy
  \begin{itemize}
    \item Short-run trade-off between inflation and the output gap; shift vertically by
    \begin{align}
       u_t
     \end{align} 
    \item Central bank has to choose whether to increase inflation, have a negative output gap, or possibly a bit of both
  \end{itemize}
  Due to shocks people can expect central bank to pursue tighter policy from $t+1$ onwards;short-run trade-off will be shifted by the change 
  \begin{align}
   u_t+E_t\pi_{t+1} 
  \end{align}
  Shift will actually be smaller and thus possibly increase stabilisation. \\
Now the main issue here of course is that it might not be practically feasible to pick a policy and stick to it.   
\end{frame}
%--------------------------------------

%--------------------------------------
\begin{frame}
  \textbf{Empirical issues}\\
  Central role for NKPC in NKM: relies on output gap $x_t$
  \begin{itemize}
    \item How to measure this gap?
  \end{itemize}
 Can assume that on average output tend to return to its natural rate
 \begin{itemize}
   \item Use simple trend as proxy for natural rate (e.g. HP-filter)
 \end{itemize}
\end{frame}
%--------------------------------------

%--------------------------------------
\begin{frame}
 Proxy
 \begin{align}
  x=y_t-y_t^n
 \end{align}
 with
\begin{align}
  \tilde{y}_t=y_t-y_t^{tr}
\end{align}
 NKPC can be estimated with data using
\begin{align}
  \pi_t = \beta E_t \pi_{t+1} + \kappa\tilde{y}_t
\end{align}
We can't observe $E_t \pi_{t+1}$ so we substitute realised $\pi_{t+1}$ and use an instrumental variable to deal with the fact this this is a noisy estimator of what we really want.
\end{frame}
%--------------------------------------

%--------------------------------------
\begin{frame}
 One issue arising here is that often we find that
  \begin{align}
    \kappa<0
  \end{align}
  Seems counterintuitive but we know that
\begin{enumerate}
  \item $\Delta\pi_t$ is negatively correlated with the unemployment rate
  \item Therefore positively correlated with the output gap
\end{enumerate}
\end{frame}
%--------------------------------------

%--------------------------------------
\begin{frame}
  Given that $\beta\approx1$, we can proxy
\begin{align}
  \pi_t-\beta E_t\pi_{t+1}
\end{align} 
with 
\begin{align}
  \pi_t-\pi_{t+1}=-\Delta\pi_{t+1}
\end{align}
Negative sign on $\kappa$ might not be that surprising: two possible reasons for failure
\begin{enumerate}
  \item The model is wrong
  \item The output gap is measured with error
\end{enumerate}
\end{frame}
%--------------------------------------



%--------------------------------------
\begin{frame}
  \textbf{Gali \& Gertler} (1999) argue that the output gap is measured with error
  \begin{itemize}
    \item Deterministic trends do a bad job in capturing movements in the natural rate of output
  \end{itemize}
  \medskip
  Suggest using unit labour costs as proxy for marginal costs
  \begin{itemize}
    \item Proxy for real marginal costs is the labour share of income.
  \end{itemize}
  Estimate 
  \begin{align}
   \pi_t = \beta E_t \pi_{t+1} + \gamma s_t    
  \end{align}
  Find
  \begin{align}
    \beta>0
  \end{align} 
\end{frame}
%--------------------------------------

%--------------------------------------
\begin{frame}
  figure
\end{frame}
%--------------------------------------

%--------------------------------------
\begin{frame}
\textbf{Rudd \& Whelan} shown that there is actually downward trend in labour share across countries\\
 If the KNPC would work well with either the labour share or another measure of real marginal cost, this would imply that it is completely forward looking. 
\begin{align}
  \pi_t = \gamma \sum_{k=0}^{\infty} \beta^k E_t s_{t+k}
\end{align}
\end{frame}
%--------------------------------------

%--------------------------------------
\begin{frame}
A VAR model could be used to forecast the levels of $s_{t+k}$ and give a fitted value for the equations above.
However, research has shown (Rudd \& Whelan, 2006) that the fits are not really good. 
In contrast, adding lagged inflation to the model
\begin{align}
  \pi_t = \gamma \sum_{k=0}^{\infty} \beta^k E_t s_{t+k} + \rho \pi_{t-1}
\end{align}
improves the fit of the model considerably. 
So a main problem with the New-Keynesian Phillips Curve is that it doesn't account properly for inflation's dependence on its own lags. 

\end{frame}
%--------------------------------------




%--------------------------------------
\end{document}
