\documentclass{beamer}
\usetheme{}
\usecolortheme{dolphin}           
\useinnertheme{circles}
\setbeamertemplate{itemize items}[default]
\setbeamertemplate{enumerate items}[default]
\usepackage[T1]{fontenc}
\usepackage[utf8]{inputenc}
\usepackage{lmodern}
\usepackage{amsmath}
\usepackage{booktabs} 
\usepackage{graphicx}        
\usepackage{array}
\usepackage{color}
\makeatletter
\def\zapcolorreset{\let\reset@color\relax\ignorespaces}
\def\colorrows#1{\noalign{\aftergroup\zapcolorreset#1}\ignorespaces}
\makeatother
\graphicspath{{/home/swl/Dropbox/ucd/advanced_macro/figures/}} 
\setbeamertemplate{navigation symbols}{}

%--------------------------------------
\title{Vector Autoregression: Examples}
\author{School of Economics, University College Dublin}
\date{Spring 2018}
\begin{document}

%--------------------------------------
\begin{frame}
 \titlepage
\end{frame}
%--------------------------------------

%--------------------------------------
\begin{frame}
  \textbf{Stock \& Watson (2001)} Effect of monetary policy shocks. \\
   VAR model can be useful from two perspectives  
   \medskip
  \begin{enumerate}
    \item Scientific
    \begin{itemize}
      \item Monetary policy co-moves with lots of other macro variables
      \item Only by identifying the structural or exogenous shocks to policy can we discover its true effects
    \end{itemize}
    \medskip
    \item Policy
    \begin{itemize}
      \item Can help answer the question "if I choose to raise interest rates by an extra quarter point today, what is likely to
happen over the next year to inflation and output relative to the case
where I keep rates unchanged?"
    \item This is basically a question about impulse responses
    \end{itemize}
  \end{enumerate}
\end{frame}
%--------------------------------------

%--------------------------------------
\begin{frame}
  Quarterly data, three variables
\begin{enumerate}
  \item inflation $\pi_t$
  \item unemployment rate $u_t$
  \item federal funds rate $i_t$
\end{enumerate}
\medskip
Lower-triangular causal chain 
\begin{align}
  AZ_t = \begin{pmatrix}
    a_{11} & 0 & 0 \\
    a_{21} & a_{22} & 0\\
    a_{31} & a_{32} & a_{33}
  \end{pmatrix}
  \begin{pmatrix}
    \pi_t \\ u_t \\ i_t
  \end{pmatrix}
  = BZ_{t-1} + \epsilon_t
\end{align}
\end{frame}
%--------------------------------------

%--------------------------------------
\begin{frame}
  \textbf{Identifying assumptions}
\begin{enumerate}
  \item Inflation depends only on lagged values of the other variables (sticky prices?)
  \item Unemployment depends on contemporaneous inflation but not the funds rate
  \item The funds rate depends on both contemporaneous inflation and unemployment
\end{enumerate}
\end{frame}
%--------------------------------------


%--------------------------------------
\begin{frame}
 \textbf{Kilian (2009)} Oil price shocks\\
\begin{enumerate}
  \item What is an oil price shock?
  \item Are there different type of shocks?
\end{enumerate}
 \medskip
 Kilian identifies three types of shock
\begin{enumerate}
  \item Supply: oil production growth rate $\Delta prod_t$
  \item Demand: global demand measured by real global economic activity $rea_t$
  \item Speculation: in oil price market, measured by real oil price $rpo_t$ 
\end{enumerate}
\end{frame}
%--------------------------------------

%--------------------------------------
\begin{frame}
\begin{align}
  z_t=(\Delta prod_t, rea_t, rpo_t)'
\end{align}

  \begin{align*}
   A_0 z_t &= \alpha + \sum_{i=1}^{24} A_i z_{t-1} + \epsilon_t\\
   A_0 &= \begin{pmatrix}
     a & 0 & 0\\
     b & c & 0\\
     d & e & f
    \end{pmatrix} 
\end{align*}
\end{frame}
%--------------------------------------

%--------------------------------------
\begin{frame}
  \textbf{Identifying assumptions}
  \begin{enumerate}
  \item Oil production does not respond within the month to world demand and oil prices
  \item World demand is affected within the month by oil production, but not by oil prices
  \item Oil prices respond immediately to oil production and world demand
\end{enumerate}
\end{frame}
%--------------------------------------

%--------------------------------------
\begin{frame}
   \textbf{Reduced-form model}
    \begin{align*}
      z_t = A_0^{-1}\alpha + A_0^{-1} \sum_{i=1}^{24} A_i z_{t-1} + A_0^{-1} \epsilon_t
    \end{align*}
  $A_0$ is lower-triangular matrix, so is $A_0^{-1}$ as well.\\
  Relation between reduced-form shocks $e_t$ and structural shocks $A_0^{-1} \epsilon_t$
    \begin{align*}
      \begin{pmatrix}       e_t^{\Delta prod} \\ e_t^{rea}  \\ e_t^{rpo}      \end{pmatrix}
      =
      \begin{pmatrix}
        a_{11} & 0 & 0 \\
        a_{21} & a_{22} & 0\\
        a_{31} & a_{32} & a_{33}
      \end{pmatrix}
      \begin{pmatrix}       \epsilon_t^{\Delta prod} \\ \epsilon_t^{rea} \\ \epsilon_t^{rpo}      \end{pmatrix}
    \end{align*}
\end{frame}
%--------------------------------------


%--------------------------------------
\begin{frame}
  Practically this entails that 
\begin{enumerate}
  \item Oil production reduced form shock is a structural shock
  \item Economic activity reduced form shock is combination of structural oil shock and structural activity shock
  \item Reduced form oil price shock is combination of all three structural shocks
\end{enumerate}

\end{frame}

\begin{frame}
  \textbf{Structural model}
\begin{align*}
  AY_t=BY_{t-1} + C\epsilon_t
\end{align*}
 \medskip
 Needs 18 identifying restrictions: $2n^2=18$
 \begin{enumerate}
   \item Assuming contemporaneous interaction between variables $C=I$ (9)
   \item Zero restrictions/lower diagonal assumtpion on $A_0$ (3)
   \item Unit coefficient normalisation on diagonal $A_0$ (3)
   \item Orthogonal structural shocks: off-diagonal elements of $\sum$ are 0 (3)
 \end{enumerate}
\end{frame}
%--------------------------------------

%--------------------------------------
\begin{frame}
  Recall Vector Moving Average representation    
\begin{align*}
  Y_t = e_t + Ae_{t-1} + A^2e_{t-2} + A^3e_{t-3} + ..... + A^te_0
\end{align*}
One can repeat this calculation three times, each time with only one type of shock turned on and the other set to zero. 
Adding these up, one will get the realized values of $Y_t$.
Alternatively, one can do a dynamic simulation of the model
\begin{align*}
  Y=AY_{t-1}+\epsilon_t
\end{align*}
Here we let $\epsilon_t$ represent one of the realised shocks, setting the others to zero
\end{frame}
%--------------------------------------

%--------------------------------------
\begin{frame}
  \textbf{Brunner (2002)} El Niño and world commodity prices
  
\end{frame}
%--------------------------------------

%--------------------------------------
\begin{frame}
  \begin{align}
    ENSO_t &= \mu_s +A_{11}(L)ENSO_{t-1} + \epsilon_t\\ \nonumber
    X_t &= \phi_s + A_{21}(L)ENSO_t + A_{22}(L)X_{t-1} + \eta_t\\
    \begin{bmatrix}      \epsilon_t \\ \eta_t    \end{bmatrix}
    &\sim N \left ( \begin{bmatrix} 0 \\ 0 \end{bmatrix},
    \begin{bmatrix} \sigma^2_{\epsilon} & 0 \\ 0 & \sum_{\eta} \end{bmatrix}
     \right)
  \end{align}
  \begin{align}
    X_t=[\pi^{cp}_t-\pi^g_t \pi^g_t \Delta y_t]
  \end{align}
\end{frame}
%--------------------------------------

%--------------------------------------
\begin{frame}
  \textbf{Identifying assumptions}
  \begin{enumerate}
    \item ENSO events are not influences by contemporaneously economic events
    \item Assumption that ENSO events are exogenous can be tested
    \item $\sum_{\eta}$ is expecte to be nondiagonal
  \end{enumerate}
\end{frame}
%--------------------------------------


\end{document}
