\documentclass{beamer}
\usetheme{}
\usecolortheme{dolphin}           
\useinnertheme{circles}
\setbeamertemplate{itemize items}[default]
\setbeamertemplate{enumerate items}[default]
\usepackage[T1]{fontenc}
\usepackage[utf8]{inputenc}
\usepackage{lmodern}
\usepackage{amsmath}
\usepackage{booktabs} 
\usepackage{graphicx}        
\usepackage{array}
\usepackage{color}
\makeatletter
\def\zapcolorreset{\let\reset@color\relax\ignorespaces}
\def\colorrows#1{\noalign{\aftergroup\zapcolorreset#1}\ignorespaces}
\makeatother
\graphicspath{{/home/swl/Dropbox/ucd/advanced_macro/figures/}} 
\setbeamertemplate{navigation symbols}{}
\setbeamertemplate{footline}[frame number]
%--------------------------------------
\title{Real Business Cycle}
\author{School of Economics, University College Dublin}
\date{Spring 2018}
\begin{document}

%--------------------------------------
\begin{frame}
 \titlepage
\end{frame}
%--------------------------------------

%--------------------------------------
\begin{frame}
  \textbf{Real Business Cycle model:} Assumes
  \begin{enumerate}
    \item Perfectly functioning competitive markets
    \item Rational expectations
  \end{enumerate}
  \medskip
  Outcomes generated by decentralized decisions of firms and households: can be replicated as solution to a social planner problem who want so maximise
 \begin{align}
  E_t \left[\sum^{\infty}_{i=0} \beta^i(U(C_{t+i})-V(N_{t+i})) \right]
 \end{align}
 $C_t$ is consumption\\
 $N_t$ hours worked\\
 $\beta$ is the household's rate of time preference 
\begin{align}
  U(C_t)-V(N_t)=\frac{C_t^{1-\eta}}{1-\eta}-\nu N_t
\end{align}
\end{frame}
%--------------------------------------

%--------------------------------------
\begin{frame}
 \textbf{Economic constraints}
\begin{align}
  Y_t &= C_t + I_t = A_tK^\alpha_{t-1}N^{1-\alpha}_t\\
  K_t &= I_t + (1-\gamma)K_{t-1}
\end{align}
  Technology process $A_t$ is usually a log-linear AR(1) process
  \begin{itemize}
    \item For simplicity assume that $A_t$ does not trend over time: economy has average growth rate of zero.
  \end{itemize}
 \begin{align}
  ln A_t= (1-\rho) ln A^* + \rho ln A_{t-1} + \epsilon_t
\end{align}
$A^*$ indicates the steady-state for technology.
\end{frame}
%--------------------------------------

%--------------------------------------
\begin{frame}
 \textbf{Solving the model}
 \begin{enumerate}
  \item Formulating the Lagrangean
  \item Finding the first order conditions (FOCs)
  \item Log-linearisation of the FOCs
  \item Finding the steady-state
\end{enumerate}
\end{frame}
%--------------------------------------

%--------------------------------------
\begin{frame}
  \textbf{Constraints}  
\begin{align}
  Y_t &= C_t + I_t = A_tK^\alpha_{t-1}N^{1-\alpha}_t\\ \nonumber
  K_t &= I_t + (1-\gamma)K_{t-1}
\end{align}
 Can combine in single equation

\begin{align}
  A_tK^\alpha_{t-1}N^{1-\alpha}_t=C_t + K_t - (1-\gamma)K_{t-1}
\end{align}
Can formulate this as a Langrangian problem

\end{frame}
%--------------------------------------

%--------------------------------------
\begin{frame} 
\begin{align}
  L &= E_t \sum^{\infty}_{i=0}\beta^i[U(C_{t+i}) - V(N_{t+i})] +\\ \nonumber
  & E_t \sum^{\infty}_{i=0}\beta^i \lambda_{t+i} [A_tK^\alpha_{t+i-1}N^{1-\alpha}_t + (1-\gamma)K_{t+i-1} - C_{t+i} - K_{t+i}]
\end{align}
 The Langrangian involves picking a series of values for consumption and labour, subject to satisfying a series of constraints. 
\end{frame}
%--------------------------------------

%--------------------------------------
\begin{frame}
  \textbf{Infinity:} Equations sums to infinity: entails infinite number of first-order conditions for current and expected values of $C_t, K_t,N_t$.
  \begin{itemize}
    \item Can be simplified by looking at when exactly the time $t$ and $t+n$ variables appear
  \end{itemize}
  NB - derivative for capital is 
  \begin{align}
    \frac{\delta L}{\delta K_t}
  \end{align}
  Find when $K_t$ appears.
\begin{align}
  U(C_t)-V(N_t)+ \lambda_t(A_tK^\alpha_{t-1}N^{1-\alpha}_t -C_t -K_t + (1-\gamma)K_{t-1}) \\ \nonumber
  + \beta E_t[\lambda_{t+1}(A_{t+1}K^\alpha_{t}N^{1-\alpha}_{t+1}+(1-\gamma)K_t)]
\end{align}
\end{frame}
%--------------------------------------

%--------------------------------------
\begin{frame}
  \begin{itemize}
  \item $t$ variables only appear once: their FOCs consist of differentiating the model end setting the derivatives equal to zero
  \item $t+n$ appear exactly as the $t$ variables, only in expectation form and multiplied by discount $\beta^n$: their FOCs are identical to the $t$ variables. 
\end{itemize}
\end{frame}

%--------------------------------------
\begin{frame}
\begin{align}
  Y_t = A_tK^\alpha_{t-1}N^{1-\alpha}_t 
  \end{align}
  Differentiating we get the following FOCs
\begin{align}
  \frac{\delta L}{\delta C_t}&: U'(C_t)-\lambda_t=0\\
  \frac{\delta L}{\delta K_t}&: -\lambda_t + \beta E_t\left[\lambda_{t+1} \left( \alpha\frac{Y_{t+1}}{K_t}+1-\gamma \right) \right] =0\\
  \frac{\delta L}{\delta N_t}&: -V'(N_t) + (1-\alpha) \lambda_t \frac{Y_t}{N_t}=0\\
  \frac{\delta L}{\delta \lambda_t}&: A_tK^\alpha_{t-1}N^{1-\alpha}_t - C_t - K_t + (1-\gamma)K_{t-1} =0
\end{align}
\end{frame}
%--------------------------------------

%--------------------------------------
\begin{frame}
 \textbf{Keynes-Ramsey condition}
 Now in order to make the system a bit easier to understand, it helps to define the marginal value of an additional unit of capital next year as
\begin{align}
  R_{t+1}&= \alpha \frac{Y_{t+1}}{K_t}+1-\gamma\\
  FOC &: \lambda_t=\beta E_t(\lambda_{t+1}R_{t+1})  
\end{align}

This can then be combined with the FOC for consumption to give
\begin{align}
  U'(C_t)= \beta E_t[U'(C_{t+1})R_{t+1}]
\end{align}
\begin{align}
  \frac{\Delta L}{\Delta C_t} &= U'(C_t)-\lambda_t=0\\
  \lambda_t &= U'(C_t)
\end{align}
\end{frame}
%--------------------------------------

%--------------------------------------
\begin{frame}
  This means that
\begin{itemize}
  \item The marginal utility of consumption must equal the marginal utility of capital
  \item And the marginal utility of capital must equal the expected value of capital at $t+1$ times the return of capital times a discount factor
\end{itemize}


The interpretation of the Keynes-Ramsey condition is that
\begin{itemize}
  \item A $\Delta$ decrease in consumption today will lead to a loss of $U'(C_t)\Delta$ in utility
  \item Invest to get $R_{t+1}\Delta$ tomorrow
  \item Which is worth $\beta E_t[U'(C_{t+1})R_{t+1}]$ in terms of today's utility.
  \item Along an optimal path, the household must be indifferent
\end{itemize}

\end{frame}
%--------------------------------------

%--------------------------------------
\begin{frame}
  \textbf{CCRA Consumption and Separable Consumption-Leisure}
  The model uses the utility function
  This formulation of the Constant Relative Risk Aversion (CRRA) utility from consumption and separate disutility from labour turns out to be necessary for the model to have a stable growth path solution.
\begin{align}
  U(C_t)-V(N_t)=\frac{C_t^{1-\eta}}{1-\eta}-\nu N_t
\end{align}

The Keynes-Ramsey condition becomes
\begin{align}
  C^{-\eta}_t=\beta E_t(C^{-\eta}_{t+1}R_{t+1})
\end{align}

And the condition for optimal worked hours becomes
\begin{align}
  -\nu +(1-\alpha)C^{-\eta}_t \frac{Y_t}{N_t} = 0
\end{align}
\begin{align}
  \frac{Y_t}{N_t} = \frac{v}{1-\alpha}C_t^{\eta}
\end{align}
\end{frame}
%--------------------------------------

%--------------------------------------
\begin{frame}
  The RBC model can be defined by six equations
\begin{enumerate}
  \item three identities describing resource constraints
  \item one definition
  \item and two FOCs describing optimal behaviour
\end{enumerate}
Process for the technology variable is \begin{align}
  ln A_t = (1-\rho) ln A^* + \rho ln A_{t-1} + \epsilon_t
\end{align}
\end{frame}
%--------------------------------------

%--------------------------------------
\begin{frame}
\begin{align}
  Y_t &= C_t +I_t\\
  Y_t &= A_tK^{\alpha}_{t-1}N^{1-\alpha}_t\\
  K_t &= I_t+(1-\gamma)K_{t-1}\\
  R_t &= \alpha \frac{Y_t}{K_{t-1}}+1-\gamma\\
  C^{-\eta}_t &= \beta E_t(C^{-\eta}_{t+1}R_{t+1})\\
  \frac{Y_t}{N_t} &= \frac{v}{1-\alpha}C^{\eta}_t
\end{align}
\end{frame}
%--------------------------------------

%--------------------------------------
\begin{frame}
 \textbf{Log-linearisation}
 Nonlinear systems can generally not be solved analytically.
The solution can be approximated however using a corresponding set of linear equations.
The idea is to use Taylor series approximation: any nonlinear function $F(x_t,y_t)$ can be approximated around any point ($x^*_t,y^*_t$) using the formula

\end{frame}
%--------------------------------------

%--------------------------------------
\begin{frame}
  If the gap between ($x_t,y_t$) and ($x^*_t,y^*_t$) is small, then terms in second and higher order powers and cross-terms will all be very small and can be ignored leaving something like
\begin{align}
  F(x_t,y_t)\approx \alpha+\beta_1x_t+\beta_2y_t
\end{align}
If we linearise around point that is far away from ($x_t,y_t$), then the approximation will not be accurate.
\end{frame}
%--------------------------------------

%--------------------------------------
\begin{frame}
  DSGE model use a particular version of this technique, by taking logs and linearise the logs of the variables around a steady-state path in which all real variables are growing at the same rate.
The steady-state path is relevant because the stochastic economy will, on
average, tend to fluctuate around the values given by this path, making the
approximation an accurate one.
This gives us a set of linear equations in the deviations of the logs of these variables from their steady-state values.
Remember that log-differences are approximately percentage deviations
\begin{align}
  ln X-ln Y \approx \frac{X-Y}{Y}
\end{align}
This approach gives us 
\begin{itemize}
  \item A system that expresses variables in terms of their percentage deviations from the steady-state paths
  \item A system of variables that can be thought of representing the business-cycle component of the model
  \item Coefficients are elasticities and IRFs are easy to interpret
  \item It is also easy to implement
\end{itemize}
\end{frame}
%--------------------------------------

%--------------------------------------
\begin{frame}
The key to the log-linearization method is that every variable can be written as
A log-deviation of a variable from its steady-state value is noted as $x_t=ln X_t - ln X^*$.
\begin{align}
  X_t=X^*\frac{X_t}{X^*}=X^* e^{x_t}
\end{align}
A first-order Taylor approximation for $e^{x_t}$ can be given by
\begin{align}
  e^{x_t}\approx1+x_t
\end{align}
Meaning that the variables can be written as
\begin{align}
  X_t \approx X^*(1+x_t)
\end{align}
Additionally we can set terms like $x_ty_t=0$ when multiplying variables since we are looking at small deviations from the steady state; multiplying these small deviations together one will get a term close to zero.
\begin{align}
  X_tY_t\approx X^*Y^*(1+x_t)(1+y_t) \approx X^*Y^*(1+x_t+y_t)
\end{align}
\end{frame}
%--------------------------------------

%--------------------------------------
\begin{frame}
  Start with
\begin{align} Y_t=C_t+I_t \end{align}

Re-write as
\begin{align} Y^*e^{y_t}=C^*e^{c_t}+I^*e^{i_t} \end{align}

Using first-order approximation this becomes
\begin{align} Y^*(1+y_t) = C^*(1+c_t) + I^*(1+i_t) \end{align}

Since the steady-state terms must obey identities so
\begin{align} Y^* = C^* + I^* \end{align}

Cancelling these terms on both sides we get
\begin{align} Y^*y_t = C^*c_t + I^*i_t \end{align}

Which we will write as
\begin{align} y_t=\frac{C^*}{Y^*}c_t+\frac{I^*}{Y^*}i_t \end{align}
\end{frame}
%--------------------------------------

%--------------------------------------
\begin{frame}
  \begin{align}Y_t=A_tK^{\alpha}_{t-1}N^{1-\alpha}_t \end{align}

This can be re-written in terms of steady-states and log-deviations as
\begin{align} Y^*e^{y_t} = (A^* e^{a_t}) (K^*)^{\alpha}e^{\alpha k_{t-1}} (N^*)^{1-\alpha}e^{(1-\alpha)n_t}\end{align}

Again, the steady-state values obey identities so that
\begin{align} Y^* = A^* (K^*)^{\alpha} (N^*)^{1-\alpha} \end{align}

Cancelling will give
\begin{align}e^{y_t}=e^{a_t}e^{\alpha k_{t-1}}e^{(1-\alpha)n_t} \end{align}

Using first-order Taylor approximation this becomes
\begin{align}
  (1+y_t)=(1+\alpha_t)(1+\alpha k_{t-1})(1+(1-\alpha)n_t)
\end{align}

Ignoring the cross-products of the log-deviations this simplifies to
\begin{align} y_t=a_t+\alpha k_{t-1} + (1-\alpha)n_t \end{align}

\end{frame}
%--------------------------------------

%--------------------------------------
\begin{frame}
  Once all the equations have been log-linearised , we have a system of seven equations of the form

\begin{align}
  y_t &= \frac{C^*}{Y^*}c_t + \frac{I^*}{Y^*}i_t\\
  y_t &= a_t + \alpha k_{t-1} + (1-\alpha)n_t\\
  k_t &= \frac{I^*}{K^*}i_t + (1-\gamma)k_{t-1}\\
  n_t &= y_t-\eta c_t\\
  c_t &= E_tc_{t+1} - \frac{1}{\eta}E_t r_{t+1}\\
  r_t &= \left(\frac{\alpha}{R^*}\frac{Y^*}{K^*} \right)(y_t-k_{t-1})\\
  a_t &= \rho a_{t-1} + \epsilon_t
\end{align}
The model used assumes that technology, the source of all long-term growth in the economy, is given by $a_t=\rho a_{t-1} + \epsilon_t$. This means that there is no trend growth in the economy, and as a result all the steady-state variables are constants.
\end{frame}
%--------------------------------------

%--------------------------------------
\begin{frame}
 Calculating the steady state:
 The log-linearised system contains three variables related to the steady-state path which needs to be calculated.
 These are $\frac{C^*}{Y^*},\frac{I^*}{Y^*},\frac{I^*}{K^*},\frac{\alpha}{R^*}\frac{Y^*}{K^*}$
We do this by taking the original non-linearized RBC system and figuring out
what things look like along a zero growth path.
We start with the steady-state interest rate which is linked to consumption behaviour via the so called Euler equation (or Keynes-Ramsey condition)In the vicinity of the steady state we have that $y_t=y_{t+1}=y^*$ so $\frac{y_t}{y_{t+1}}=1$
\end{frame}
%--------------------------------------

%--------------------------------------
\begin{frame}
\begin{align}
  C_t^{-\eta} &= \beta E_t(C_{t+1}^{-\eta}R_{t+1})\\
  1 &= \beta E_t \left( \left(\frac{C_t}{C_{t+1}} \right)^\eta R_{t+1} \right)
\end{align}

Because we have no trend growth in technology in our model, the steady-state features consumption, investment, and output will all take constant values with no uncertainty.
In steady-state we have
\begin{align}
  C^*_t &= C^*_{t+1}=C^*\\
  R^* &= \beta^{-1}
\end{align}
In a no-growth economy, the rate of return on capital is determined by the
rate of time preference.

\end{frame}
%--------------------------------------

%--------------------------------------
\begin{frame}
   Let's look at the rate of return on capital
\begin{align}
  R_t=\alpha \frac{Y_t}{K_{t-1}}+1-\gamma
\end{align}

In steady-state we have
\begin{align}
  R^*= \beta^{-1} = \alpha \frac{Y^*}{K^*}+1-\gamma
\end{align}

So we get
\begin{align}
  \frac{Y^*}{K^*}=\frac{\beta^{-1}+\gamma-1}{\alpha}
\end{align}

Together with the steady-state interest equation this tells us that
\begin{align}
  \frac{\alpha}{R^*}\frac{Y^*}{K^*} &=\alpha \beta \left(\frac{\beta^{-1}+\gamma-1}{\alpha} \right)\\
  &= 1-\beta(1-\gamma)
\end{align}
\end{frame}
%--------------------------------------

%--------------------------------------
\begin{frame}
  Now we only have to find the ratios for 
\begin{itemize}
  \item investment-capital
  \item investment-output
\end{itemize}
Here we can use the identity
\begin{align} K_t=I_t+(1-\gamma)K_{t-1} \end{align}
\begin{align}
  K^* &= I^* + (1-\gamma)K^*\\
  K^* &= I^* + K^* - \gamma K^*\\
  I^* &= \gamma K^*\\
  \frac{I^*}{K^*} &= \gamma
\end{align}


\end{frame}
%--------------------------------------

%--------------------------------------
\begin{frame}
This identity is in steady-state and combined with the fact that $K^*_t=K^*_{t-1}=K^*$ we get
\begin{align} \frac{I^*}{K^*}=\gamma \end{align}


This can be combined with the previous steady-state ratio to give
$\frac{Y^*}{K^*}=\frac{\beta^{-1}+\gamma-1}{\alpha}$
\begin{align}
  \frac{I^*}{Y^*}=\frac{\frac{I^*}{K^*}}{\frac{Y^*}{K^*}}=\frac{\alpha \gamma}{\beta^{-1}+\gamma-1}
\end{align}

From this it follows that the consumption-output ratio must be
\begin{align}
  \frac{C^*}{Y^*}=1-\frac{\alpha \gamma}{\beta^{-1}+\gamma-1}
\end{align}
\end{frame}
%--------------------------------------

%--------------------------------------
\begin{frame}
  \begin{align}
  y_t &= \left(1-\frac{\alpha \gamma}{\beta^{-1}+\gamma -1}\right)c_t +
  \left(\frac{\alpha \gamma}{\beta^{-1}+\gamma-1}\right)i_t\\
  y_t &= a_t +\alpha k_{t-1} + (1-\alpha)n_t\\
  k_t &= \gamma i_t + (1-\gamma)k_{t-1}\\
  n_t &= y_t-\eta c_t\\
  c_t &= E_t c_{t+1} - \frac{1}{\eta}E_t r_{t+1}\\
  r_t &= (1-\beta(1-\gamma))(y_t-k_{t-1})\\
  a_t &= \rho a_{t-1} + \epsilon_t
\end{align}
Once we make assumptions about the underlying parameter values a solution algorithm such as the Binder-Pesaran program can be used to obtain the reduced-form solution and simulate the model. 
\end{frame}
%--------------------------------------

%--------------------------------------
\begin{frame}
\begin{enumerate}
  \item Perfect markets and rational expectations
  \begin{itemize}
    \item Markets are not always competitive and people are not always rational (in their economic decisions)
    \item RBC model should be seen as a benchmark against which more complicated models can be assessed. 
    \item Separate modeling of the decisions of firms and households to account for imperfect competition can be done    
  \end{itemize}

  \item Monetary and fiscal policy
  \begin{itemize}
    \item RBC models exhibit complete monetary neutrality, so there is no role at all for monetary policy, something which many people think is crucial to understanding the macroeconomy
    \item Most models build on the RBC approach introducing mechanisms that are allowed to have Keynesian effects, such as sticky prices and
    wages 
  \end{itemize}
  \item Skepticism about technology shocks
  \begin{itemize}
    \item RBC models give primacy to technology shocks as the source of economic fluctuations (all variables apart from $A_t$ are
    deterministic). But what are these shocks?
    \item Link between long-term growth and TFP
  \end{itemize}
\end{enumerate}

\end{frame}
%--------------------------------------

%--------------------------------------
\begin{frame}
  Can check the parameterizing of the the model and simulate and check the impluse response functions. 
The following graphs are based on a model with parameter values intended for the analysis of quarterly time series
\begin{align}
  \alpha &=\frac{1}{3}\\
  \beta &=0.99\\
  \gamma &=0.015\\
  \rho &= 0.95\\
  \eta &= 1
\end{align}
\end{frame}
%--------------------------------------

%--------------------------------------
\begin{frame}
  Figure shows a 200-period simulation of the model and illustrates the main feature of the RBC model, namely that it can generate business cycles that don't look too far-fetched. 
We can notice two things here
\begin{enumerate}
  \item The model roughly matches the observed fluctuations in output
  \item The model reflects the fact that investment cycles are more volatile than consumption
\end{enumerate}
\end{frame}
%--------------------------------------

%--------------------------------------
\begin{frame}
  Part of the early hype surrounding RBC models stemmed from the idea that the model contained important propagation mechanisms turning technology shocks into business cycles. 
The idea behind this is that increases in technology would lead to extra output through higher capital accumulation and by inducing people to work more. 
This entails, as suggested in early research, that in a world with identical technology level one would expect the RBC model still to generate business cycles. 
However, these propagation mechanisms are quite weak as shown in figure which illustrates that fluctuations in output follow fluctuations in technology quite closely.
\end{frame}
%--------------------------------------

%--------------------------------------
\begin{frame}

\begin{align*}
  F(x_t,y_t) = F(x_t^*,y^*_t) +\\ 
  F_x(x^*_t,y^*_t)(x_t-x^*_t) + \\
  F_y(x^*_t,y^*_t)(y_t-y^*_t) +\\
  F_{xx}(x^*_t,y^*_t)(x_t- x^*_t)^2+ 
  F_{xy}(x^*_t,y^*_t)(x_t−x^*_t) (y_t-y^*_t)+\\
  F_{yy}(x^*_t,y^*_t) (y_t-y^*_t) +...
\end{align*}

\end{frame}
%--------------------------------------


%--------------------------------------
\end{document}
